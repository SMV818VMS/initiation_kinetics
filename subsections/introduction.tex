Introduction

First study looked on average translocation coefficients. Does not capture
dynamic effects.

Many models sequence inspecific -- we need sequence specific model.

Questions we'd like to ask:

* Can we reproduce the abortive patterns from Hsu et al.?

* Do results improve when taking into account the steric clash at ~6nt?
    -- meaning an increase in abortive initation above what is predicted by
    -- steric clash reduces forward translocation rate?
    -- DG100 and DG400 have interesting common abortive patterns for the
    datasets as a whole up to +7 at least
    -- You clearly see the effect of the +6 steric clash, so it would be fun
       to incorporate it in such a way as to 
    -- The trend is as follows High for 2-3, then low, then high for +6, then
    low until high for 13, 14 and 15. That seems to be the general abortive
    picture. Can you work with that?
    -- BUT for the lower half of DG100, the difference is most clear at +13,
    +14, and +15

    -- Hypothesis. The N25 promoter has a lot of initiation; easy open complex
    formation. The most striking part of the difference between top half and
    lower half in DG100 is the abortive probability at +13, +14, and +15. This
    abortive probability is affected by the propensity to pause - AND -
    backtrack from that paused state at these positions. If we can reproduce
    these processes, we should be able to explain PY well.

* Can we estimate the ``true'' extent of abortive initiation, given that
  full-length transcripts may be lower in abundance due to stalling at the end of
  the blunt DNA?

* Is there any effect of an oscillating trigger loop?

* Is there an effect of the RNA-DNA hybrid on backtracking rate?

* Is there an effect of the RNA-DNA hybrid on the probability to release RNA?

* Can we estimate the effect of the scrunched DNA bubble on the unstability?

* Is the combined effect of an RNA-DNA stabilizing hybrid (sequence-specific)
  and a scrunched bubble (sequence-unspecific) leading to improvements?
