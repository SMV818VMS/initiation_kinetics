Title: ``Kinetics of abortive transcription initiation''

Method: We simulate initial transcription. We assume that initially all RNAP
start in the closed complex formation with an initiating 2nt RNA. We then
follow transcribing complexes.

In reality, transcription initiates at millions of DNA-templates. RNAP binds,
formes open and closed complexes, and initiates transcription. Some abort very
fast, some transcribe a bit; backtrack; and abort; some transcribe alll the way
to promoter escape. If an RNAP backtracks, it is assumed free to start again.
BUT! Two RNAP cannot transcribe the same DNA at the same time. Would be nice to
know how long one round of initial transcription takes compared to the duration
of the experiment.

We simulate one round of transcription initiation, assuming that this
represents millions of RNAP initiating simultaneously on millions of DNA. How
long do we let the simulation run? Well, 10 minutes, but for that we need some
semi-accurate rate constants, but that should be OK I think. What is the effect
of not allowing re-synthesis? Will we under-estimate the amount of early
abortive product? If we try to fit to the abortive probability, we are actually
fitting against a different process than what is really happening. Is there any
way to transition from an abortive state to an initiating state, while at the
same time keeping track of the amount of abortive RNA? I think so. You just
need to integrate the values in the abortive states.

Another case to consider is that after FL production, the RNAP is free to
reassociate with a new promoter.


* Do a mini-review on the kinetics of elongation and initiation.

Questions we'd like to ask:

* Can we reproduce the abortive patterns from Hsu et al.?
    Consider time: did her system stabilize?

* If we can, can we estimate the effect of GreB by estimating a ``cleavage'' rate?

* Can we reproduce the RNAP distributions of Margeat et al.? 

* Do results improve when taking into account the steric clash at ~6nt?
    -- meaning an increase in abortive initation above what is predicted by
    -- steric clash reduces forward translocation rate?
    -- DG100 and DG400 have interesting common abortive patterns for the
    datasets as a whole up to +7 at least
    -- You clearly see the effect of the +6 steric clash, so it would be fun
       to incorporate it in such a way as to 
    -- The trend is as follows High for 2-3, then low, then high for +6, then
    low until high for 13, 14 and 15. That seems to be the general abortive
    picture. Can you work with that?
    -- BUT for the lower half of DG100, the difference is most clear at +13,
    +14, and +15

    -- Hypothesis. The N25 promoter has a lot of initiation; easy open complex
    formation. The most striking part of the difference between top half and
    lower half in DG100 is the abortive probability at +13, +14, and +15. This
    abortive probability is affected by the propensity to pause - AND -
    backtrack from that paused state at these positions. If we can reproduce
    these processes, we should be able to explain PY well.

* Can we estimate the ``true'' extent of abortive initiation, given that
  full-length transcripts may be lower in abundance due to stalling at the end of
  the blunt DNA?

* Is there any effect of an oscillating trigger loop?

* Is there an effect of the RNA-DNA hybrid on backtracking rate?

* Is there an effect of the RNA-DNA hybrid on the probability to release RNA?

* Can we estimate the effect of the scrunched DNA bubble on the unstability?

* Is the combined effect of an RNA-DNA stabilizing hybrid (sequence-specific)
  and a scrunched bubble (sequence-unspecific) leading to improvements?

* Test of approach: we will fit each ITS with parameter estimation, and compare
  the different estimates for the different k values.
