Different correlation between nucleobases adenine and guanine with PY. This
can be result 1 in your work. Strong correlation with number of purines. Has
been shown ... G..., butstronger with A than with G. Further, no correlation
between A and G; if A and G were equally beneficial to avoid paused and
backtracked state, one would expect these two values to have some correlation
since they would be equally likely to appear in the high-PY variants, and
equally likely not to appear in the low-PY variants.

This indicates that there is a differential contribution of these two purines
to translocation, pausing, and backtracking.

The bulk-transcription experiments of Hsu do not reveal the dynamics of the
transcription process, from which one can extract rate parameters from a
dynamic model. Nevertheless, we are going to write a kinetic model. But more
interested in general characteristics of the system, like from the questions
below.

Can you show some basic truths about the system? For example:

* does the rate of backtracking depend on the extent of transcription?
    - is backtracking with a full RNA-DNA hybrid the same speed as for a short
    RNA-DNA hybrid?
    - do we have to introduce an effect of scrunching on the rate of backtracking?

* can we model the steric clash with nascent RNA in terms of forward
    translocation difficulty

------------------------

Framework: kinetc model with the following processes:

Backward translocation
Forward translocation
Paused state (elemental? Or just paused)
Backtrack
Abort
Escape

Use the rate constants adopted from Malinen. For reverse translocation, scale
with Hein et al. somehow.

------------------------

